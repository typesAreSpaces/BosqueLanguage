%
\documentclass[runningheads]{llncs}
\usepackage{graphicx}
%
\begin{document}
% Title
\title{Automatic verification for the Bosque Programming Language}
%\thanks{} % If we want to thank someone/org.
%
% Authors
\author{Mark Marron$^1$,
  Jose Abel Castellanos Joo$^2$}
%
\authorrunning{M. Marron and J.A. Castellanos Joo}
%
\institute{{$^1$Microsoft Research\\} {$^2$University of New Mexico}}
%
\maketitle
%
\begin{abstract}  
  Automatic theorem proving tools has become extremely popular thanks to their application in software
  analysis/verification. It provides formal guarantees about software correctness and useful mechanisms
  for debugging highly non-trivial properties. However, these tools
  require a large amount of knowledge from their users to be able to use them correctly. Hence, it is
  complicated for many programmers to use verification in their projects.
  
  Bosque is a new programming language designed for the future of Cloud and IoT application development
  with the intention to simplify the verification experience for application developers.
  This paper discusses the Bosque language features of that allow the automatic verification of its
  properties using a formal encoding in \{FStar, SMTLib\}.
%
\keywords{Formal Verification \and Language Design \and Software Engineering.}
\end{abstract}
%
%
%

\section{Introduction}

%
%

\section{Background}

%
%

\section{Encoding}

Bosque is a strongly-typed functional programming language supporting nominal and
structural typing. Bosque has a small core library consisting of booleans, integers,
typed strings, tuples, records, unions, and structure-like composite data types called
concepts (non-instantiable types) and entities (instantable types).

BEGIN META\{Things to have in mind\}

\begin{itemize}
\item High level verifiable language
  \begin{itemize}
  \item Semantics
  \item Encoding
  \end{itemize}


\item Collections / Inductive data types
  \begin{itemize}
  \item Quantifiers
  \end{itemize}
\end{itemize}

END META

The intermediate representation of Bosque 

\subsection{Encoding Core Library}
\subsection{Encoding User-defined }
%
%

\section{Evaluation}

%
%

\section{Conclusion}

%
%

\bibliography{bibliography/references}
\bibliographystyle{unsrt}

\end{document}